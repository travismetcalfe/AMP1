\input rep12
\input easyng
\input ruled
\input btxmac
\nopagenumbers
\maketoc

\centerline{\largebf Notes on Opacity-Interpolationroutine OPINT 
                     (v11, 05/03/2006)}

\subsectionskip
\section{1. Introduction}
This package interpolates opacities in $\rho$,$T$,$X$ and $Z$ ($\rho$ = density,
$T$ = temperature, $X$ = hydrogen mass fraction, $Z$ = heavy elements mass fraction).
In this package the latest OPAL-tables~\cite{rog95} are used.
One can choose between two interpolation algorithms.
The first algorithm (minimum norm) uses a C$^1$ interpolant defined in 
piecewise fashion
over triangles in the $\rho--T$ plane~\cite{acm677}~\cite{niel83}.
The global interpolant is constucted by a nine paramter version of a
finite interpolation scheme introduced by Nielson~\cite{niel80}.
The second algorithm uses birational splines and is implemented
according to Helmuth Spath~\cite{Spa91}.
Interpolation in $X$ and $Z$ is performed using Akima's univariate interpolation
scheme~\cite{acm433,acm697}.
For the low-temperature opacites, tables from 
Alexander \& Ferguson (1994, ApJ 437, 879) [AF94] and from
Alexander \& Ferguson (2005, ApJ 623, 585) [AF05] are available.

Electron conduction (EC) according to Itoh et al. (1983, ApJ 273, 774) can be 
enabled for all tables. EC could be important in more massive stars.
If EC is enabled, the EC becomes effective (i.e. it
is 'switched' on) if the EC contribution modifies the resulting opcacity
by $\ge 10^{-5}$. This treshold can be changed in the code (dopints.f,
dopintc.f, dopintf.f) by changing the currently defined parameter {\bf econ=5.0d0}
to any other value.
\medskip
\section{2. Installation}
\vskip-2mm
\subsection{2.1 Using the provided opacity tables}
\vskip-2mm
The following five tables are provided for 'little\_endian' machines (such as 
machines with intel processors; to generate tables for `big\_endian' machines 
see Section~2.2; for the opacity table 1), given below, also
`big\_endian' binary files are provided):
\bigskip
{\small
%\TightTables
\LeftJustifyTables
\ruledtable
 Opacity table  |Abundance      |       Description\cr
1) OPAL95+AF94  |GN93           | \parasize=7.2cm
                                  \para{Opal (1995) tables plus Alexander-Furgenson (1994)
                                     tables using the Grevesse-Noel (1993) abundances.}\nr
2) OPAL95+AF05  |GN93           | \parasize=7.2cm
                                  \para{Opal (1995) tables plus Alexander-Furgenson (2005)
                                     tables using the Grevesse-Noel (1993) abundances.}\nr
3) OPAL95+AF05  |AGS05          | \parasize=7.2cm
                                  \para{Opal (1995) tables plus Alexander-Furgenson (1994)
                                     tables using the Asplund-Grevesse-Sauval (2005)
                                     abundances.}\nr 
4) OPAL95+AF05  |AGS05pNe45Ar40 | \parasize=7.2cm
                                  \para{As table 3), but with Neon enhanced by 0.45 dex and
                                     Argon by 0.40 dex.}\nr
5) OPAL95+AF05  |AGS05pNe40     | \parasize=7.2cm
                                  \para{As table 3), but with Neon enhanced by 0.50 dex.}
\endruledtable
\bigskip
\centerline{{\bf Table 1:} Provided opacity tables}
}%\small
\vfill\eject
To use any of the tables listed above, copy in the main-directory the appropriate 
Makefile\_*, listed in the table below, to the 
filename `Makefile', e.g.: to use table 3) 
(OPAL95+AF05 and AGS05 abundances) say:
\smallskip
\line{\quad cp Makefile\_AGS05 Makefile\hfil} 
\bigskip
\bigskip
{\small
\LeftJustifyTables
\ruledtable
     Table (Abundance)       |       Makefile            |  OPINTPATH          \cr
OPAL95+AF94 (GN93)           |  Makefile\_GN93\_AF94     |  OPINTPATH\_GN93\_AF94\nr
OPAL95+AF05 (GN93)           |  Makefile\_GN93           |  OPINTPATH\_GN93     \nr
OPAL95+AF05 (AGS05)          |  Makefile\_AGS05          |  OPINTPATH\_AGS05    \nr
OPAL95+AF05 (AGS05pNe45Ar40) |  Makefile\_AGS05pNe45Ar40 |  OPINTPATH\_AGS05    \nr
OPAL95+AF05 (AGS05pNe40)     |  Makefile\_AGS05pNe50     |  OPINTPATH\_AGS05
\endruledtable
}%\small
\bigskip
\centerline{{\bf Table 2:} Provided Makefiles and OPINTPATHs}
\bigskip
Next modify the {\bf Makefile} with an editor (such as vi) according to your
compiler needs, such as compiler-flages, ranlib, etc and execute the command
\medskip
\line{\quad make lib\hfil}
\medskip
which builds the library: `./lib/libopint.a'.\hfil\break
Next copy the appropriate OPINTPATH\_* file to $<$appl\_path$>$/OPINTPATH, where
$<$appl\_path$>$ is the directory name of your application for which you wish to
use this interpolation package, e.g.:
\medskip
\line{\quad cp OPINTPATH\_AGS05 $<$appl\_path$>$/OPINTPATH\hfil}
\medskip
With an editor (such as vi) modify the pathes
of the opacity tables in the file {\bf OPINTPATH}, i.e. change the pathes
to the (e.g., absolute) installation pathes $<$inst\_path$>$ where you installed the
opacity package, as demonstrated below in Table~3 for the OPAL95+AF05 (AGS05)
opacity tables.
\medskip
{\small
\TightTables
\ruledtable
Entry \#|pathes of opacity tables                                | description        \cr
1|$<$inst\_path$>$/v11/ol95\_AGS05/little\_endian/opalxe.bin\hfill|\parasize=4.1cm
                                                              \para{Opacity values of 
                                                                    OPAL tables}       \cr
2|$<$inst\_path$>$/v11/af05\_AGS04/little\_endian/af05xe.bin\hfill|\parasize=4.1cm
                                                              \para{Opacity values of
                                                                    Alexander-Furgenson
                                                                    tables}       \cr
3|$<$inst\_path$>$/v11/ol95\_AGS05/little\_endian/opalxeAF05-pd.bin\hfill|\parasize=4.1cm
                                                              \para{Partial derivatives 
                                                                of OPAL and 
                                                                Alexander-Furgenson
                                                                tables}                \cr
4|$<$inst\_path$>$/v11/ol95\_AGS05/little\_endian/ival95.dat\hfill|\parasize=4.1cm
                                                              \para{used for determinig
                                                                    extrapolation domain 
                                                                    of opacity tables}
\endruledtable
}%\small
\medskip
\centerline{{\bf Table 3:} Format of OPINTPATH defining the input filenames 
            of the opacity tables}
%\vfill\eject
\bigskip
\subsection{2.2 Installing new OPAL opacity tables}
The package allows the use of OPAL tables of any chemical mixture (abundances) for the 
heavy elements as provided by the OPAL Web-page:
\medskip
\line{\quad http://www-phys.llnl.gov/Research/OPAL/type1inp.html\hfil} 
\medskip
After completing the form on this Web-page and pressing the ``NORMALIZE'' button, you 
will be asked for your name and e-mail address. You will be notified by e-mail from 
where you can dowload the new OPAL tables, which will usually have a filename like:
\medskip 
\line{yyyymmdd\#\#\#\#.tab\hfil}
\medskip 
where `yyyymmdd' represents the date and `\#\#\#\#' is a running number (e.g. 0001). This file
must be stored as an ASCII file (and not as an HTML file when downloaded with an internet
browser like Netscape). The procedure to prepare this new table (actually tables) for the
opint package is as follows:
\itemize{1cm}
\item{1)} go to the main directory of the opacity interpolation package.
\item{2)} start the setup progamme
\itemitem{\qquad} {\bf ./setup.sh}
\item{\phantom{2}} where you will be asked for the filename (including the complete path) 
                   of the newly downloaded opacity table 
                   ($<$table\_path$>$/yyyymmdd\#\#\#\#.tab),
\item{\phantom{2}} and for the binary format: `little' or `big' ENDIAN (little 
                   is default).  The `ENDIAN' input is only used for saving 
                   the binary tables in the subdirectory `little\_endian' or 
                   `big\_endian' located in the directory `ol95\_$<$name$>$'
                   (see below). One has still to define the actual binary 
                   format by specifying the appropriate compiler option in the 
                   Makefile (or it is determined solely by the machine 
                   architecture, see also below).
\item{\phantom{2}} Next you will be asked for a $<$name$>$ that will be used 
                   to create the new directory `ol95\_$<$name$>$', where 
                   the new tables will be copied to (e.g. $<$name$>$ could be 
                   a string that indicates the chemical composition 
                   (abundances), such as `AGS05pNe35').
\item{\phantom{2}} Finally one has to choose between two (existing) 
                   low-temperature AF05 opacity tables of different chemical 
                   compositions: AGS04 (default) or GN93, where AGS04 is the 
                   Asplund, Grevesse \& Noel (2004) composition and GN93 is 
                   the Grevesse \& Noel (1993) abundances. 
\item{3)} Edit the Makefile (compiler options, etc.) followed by
\itemitem{} make tab
\enditemize
\bigskip
Once the `make'-run has (successfully) finished, you will find the appropriate 
OPINTPATH file for the newly generated tables in the main directory 
of the opint package.

A similar setup programme exists, {\bf ./setup\_bigendian.sh}, that allows one to 
remake the provided opacity tables 2-5 (see Table~1 above) in 
`big' ENDIAN format on an appropriate `big' ENDIAN machine or with appropriate 
compiler options (e.g. `-convert big\_endian' with the Intel Fortran compiler).
\vskip 8mm
\section{3. Using the package}
Three subroutines (s/r) have to be called: 1.) {\bf maceps}, 2.) {\bf opinit} 
and 3.) {\bf opintf} or {\bf opintc} and/or {\bf opints}. The first s/r
({\bf maceps}) evaluates the relative machine precision, 
s/r {\bf opinit} initializes the opacity-tables i.e., it reads
all tables into the machine's memory.  The third s/r {\bf opint$\{\bf cfs\}$} 
is(are) the actual interpolation-routine(s) (see below).
\bigskip
\hrule
\example
\obeyspaces{
      implicit double precision (a-h,o-z)
c
c     test-driver program for opacity
c     interpolation subprograms
c     opinit       (opacity initialisation)
c     opintf       (opacity interpolation minimum norm)
c     opints       (opacity interpolation rational splines)
c
      character*80 tabnam
c
      data tabnam /'OPINTPATH'/
      data iorder /4/
      data imode  /2/ ! initialize rat. splines \& enables EC
c                     ! use imode /-2/ to disable EC
c
c     get relative machine precision
     {\bf call maceps(eps)}
c
c     initialize opacity tables
     {\bf call opinit(eps,iorder,tabnam,imode)}
c

}
\endexample
\bigskip
\hrule
\bigskip
\centerline{{\bf Figure 1:} Example how to initialize OPINT (excerpt of doptesf.f)}
\vskip 1cm
The argument variable {\bf iorder} in call {\bf opinit} defines how many 
table-points should be used
for the univariate interpolation in the X-- and Z-- domain respectively.
This number defines the degree of the used univariate polynomial~\cite{acm697}. 
Interpolation in Z is performed logarithmical and linear in X.
With iorder=4 five tables will be used for the univariate interpolation in
Z and X and will provide the most accurate interpolant (and the most expensive
in computation time).
\vskip 8mm
The initialisation subroutine {\bf opinit} has the following arguments:
\vskip 8mm

\TightTables
\ruledtable
argument|    type          | inp/out |  description                    \cr
eps     | double precision | input   |\parasize=7.0cm
                                      \para{used to evaluate certain
                                            machine-depend tolerances;
                                            evaluated from s/r maceps} \cr
iorder  | integer          | input   |\parasize=7.0cm
                                      \para{iorder+1 X and Z table-points
                                            will be used for the univariate 
                                            interpolation in the X and Z 
                                            domain, respectively\hfill\break
                                            $2\le$ iorder $\le 4$}       \cr
tabnam  | character*(130)  | input   |\parasize=7.3cm
                                      \para{defines the filename, which
                                            opacity-tables should be used;
                                            take care of the sequence 
                                            as given below in Table 5}  \cr
imode   | integer          | input   |\parasize=7.3cm
                                      \para{selects algorithm(s) to be 
                                            used;
                                            \itemize{2.0cm}
                                            \item{imode=0,1:} only the minimum norm
                                            \hfill\break
                                            algorithm will be initialized
                                            (EC is enabled with imode=1)
                                            \item{imode=2:} only the birational splines
                                            are available (EC enabled)
                                             \item{imode$>$2:} both algorithm, the
                                            minimum norm and birational splines (+EC)
                                            are initialized (see also Table 4)
                                            \item{imode$\le$0:} a negative value of
                                            the particular imode number disables
                                            the contributions due to electron conduction
                                            \enditemize\vskip -6mm}
                                            
\endruledtable
\bigskip
\centerline{{\bf Table 4:} argument description of routine opinit}
%\vskip 8mm
%\vfill\eject

\bigskip
There are three interpolation-routines available, which have to be  selected
(initialized) properly by the argument variable {\bf imode} in the previous 
call {\bf opinit}
(see also Table 4):
\bigskip
\TightTables
\ruledtable
imode |      interpolation-routine     |    description                                \cr
1     |         opintf                 |\parasize=8.3cm
                                        \para{minimum norm fast version {\bf without} 
                                               extrapolation-domain checking + EC}     \cr
1     |         opintc                 |\parasize=8.3cm
                                        \para{minimum norm version {\bf with} 
                                              extrapolation-domain\hfill\break
                                               checking + EC}                          \cr
2     |         opints                 |\parasize=8.3cm
                                        \para{birational splines {\bf with} 
                                              extrapolation-domain checking + EC}      \cr
$\le$0|    Disables EC                 |\parasize=8.3cm
                                        \para{Electron conduction (EC) is 
                                              {\bf disabled} with a negative value of the
                                              particular imode number (e.g., imode=
                                              \hbox{-1} initializes either opintf or opintc
                                              and disables electron conduction}
\endruledtable
\bigskip
\centerline{{\bf Table 5:} available interpolation subroutine calls}
\bigskip
%\vfill\eject
The following listing is an excerpt of the programme doptesf.f:
\bigskip
\hrule
\example
\obeyspaces{
c
c    interpolate to get opacity value opalg
c    iexp=0
     ier=0
    {\bf call opintf(x,z,tlg,rlg,opalg,opr,opt,opx,opz,iexp,ier)}
     if(ier.gt.0)write(6,'(a,i3)')
   +   'opint: ERROR in interpolation s/r, ier=',ier
c    if(iexp.gt.0)print *,'nr. of extrapolated points: ',iexp
c
}
\endexample
\bigskip
\hrule
\bigskip
\centerline{{\bf Figure 2:} Example how to use routine opintc}
\vskip 5truemm
For the extrapolation domain checking of the OPAL-tables use 
{\bf opintc} instead of {\bf opintf} and uncomment:
\bigskip
\example\obeyspaces{
c    iexp=0  
c    if(iexp.gt.0)print *,'extrapolated points: ',iexp 
}
\endexample
\vskip 8mm
The OPAL95 tables do not provide values for the region
7.2 $\le\log_{10}(T) \le$ 8.7 and -0.5 $\le$ $\log_{10}(R)$ $\le$ 1.0.
This domain (here called the extrapolation domain) has been reconstructed by 
a proper extrapolation scheme using the univariate Akima interpolation scheme.
The argument variable {\bf iexp} indicates how many of these extrapolated
table-points have been used in the interpolation process.
Moreover, in this version the range of $\log_{10}(R)$ has been extended to
$\log_{10}(R)=5$, using linear extrapolation in $\log_{10}(R)$. This will
allow an (very) approximate estimate of the opacity for low-mass stars.
\bigskip
The argumentlist is the same for all three routines opintf, opintc and opints:
\sectionskip

\TightTables
\ruledtable
argument|    type          | inp/out |  description                        \cr
x       | double precision | input   |\parasize=6.0cm
                                      \para{hydrogen mass fraction X\hfill\break
                                            $0.0\le$x$\le 0.9$}            \cr
z       | double precision | input   |\parasize=6.0cm
                                      \para{mass fraction of heavy elements Z
                                            \hfill\break $0.0\le$z$\le0.1$}\cr
tlg     | double precision | input   |\parasize=6.0cm
                                      \para{logarithm of base 10 of temperature $T$
                                      \hfill\break$\{3.0\} 2.70\le$tlg$\le 8.7$}\cr
rlg     | double precision | input   |\parasize=6.0cm
                                      \para{logarithm of base 10
                                            $\rho \over T6^3$ ; 
                                            $\rho = $ density;
                                            $T6 = {T \over 10^6}$;
                                            \qquad $-8.\le$rlg$\le 5.$
                                           }                               \cr
opalg   | double precision | output  |\parasize=6.0cm
                                      \para{logarithm fo base 10 of opacity 
                                            $\kappa$}\cr
opr     | double precision | output  |\parasize=6.0cm
                                      \para{partial derivative 
                                            ${\partial(\log\kappa)\over
                                             \partial(\log\rho)}_{T,X,Z}$}     \cr
opt     | double precision | output  |\parasize=6.0cm
                                      \para{partial derivative 
                                            ${\partial(\log\kappa)\over
                                             \partial(\log T)}_{\rho,X,Z}$}     \cr
opx     | double precision | output  |\parasize=6.0cm
                                      \para{partial derivative
                                            ${\partial(\log\kappa)\over
                                             \partial(\log X)}_{\rho,T,Z}$}     \cr
opz     | double precision | output  |\parasize=6.0cm
                                      \para{partial derivative
                                            ${\partial(\log\kappa)\over
                                             \partial(\log Z)}_{\rho,T,X}$}     \cr
iexp    | integer          | output  |\parasize=6.0cm
                                      \para{\itemize 1.5cm
                                            \item{opint[cs]:} \# of table points
                                               lying in the extrapolation domain
                                            \item{opintf \ \ :} 0 
                                            \enditemize\vskip -6mm}          \cr
ier     | integer          | output  |\parasize=6.0cm
                                      \para{if ier $=$ 1, the point lies
                                            outside the given table domain, and
                                            will be estimated by extrapolation
                                            using Shepard's method; otherwise
                                            ier=0}

\endruledtable
\bigskip
\itemize 1.8truecm
\litem{\bf Table 6:} argument description of routine opintf (opintc, opints).
                     Curly brackets indicate values for AF94 tables.
\enditemize
\vfill\eject
\section{4. Linking the interpolation routines with the main programme}
After successful compilation of the OPINT-routines, the link editor
under UNIX may be called as
\bigskip
\centerline{\small{\$(F77) \$(FFLAGS) {\it your\_programme}\quad -L{\it your\_path}/v9/lib\quad 
                                        -lopint\quad }}
%\vfill\eject
\bigskip
\section{5. Common-blocks}
The following common blocks must not be altered between s/r call opinit and s/r 
opintf (opintc, opints):
\bigskip
\hrule
\example
\obeyspaces{
c
c     pointers for the s/r opintd (used as argument 
c     for s/r masubi)
      common /jpoint/ nti
      common /ipoint/ iali
c
c     partial derivatives and triangulation indices
      common /pderiv/ iwk(niwkl),wk(nwkl,ntab,nzva)
c
      common /opadat/ opa(nxir,nyir,ntab,nzva),
     +                rlg(nxir,ntab,nzva),
     +                tlg(nyir,ntab,nzva)
      common /valdat/ ivalo(nyir,ntab,nzva)
      common /tablex/ xtab(ntab),iorder
      common /tablez/ ztab(nzva)
      common /xyzdat/ xd(ndat,nzva),
     +                yd(ndat,nzva),
     +                zd(ndat,ntab,nzva)
      common /machin/ drelpr,toll,eps10
c
c     birational spline coefficients
      common /birasp/ pp,tls(nyif),
     +                ra(nxir,nyif,4,4,ntab,nzva)
c     table dimension for s/r opintc{f} and opints
      common /tabdim/ nzvai,ntabi,nxiri,nyiri,nyisi,iyifi,mdi,
     +                nti,iali
}
\endexample
\bigskip
\hrule
\bigskip
\centerline{{\bf Figure 3:} common-blocks used in the OPINT-package}
\bigskip
\bigskip

\section{6. File unit numbers}
The following logical unit numbers are use in subroutine ``opinit''
in open-statements:

$lun = 29,\ 31,\ 32,\ 33,$
\medskip
and should therefore not be used in the calling routines.
%\vfill\eject
\vskip 8mm
\leftline{\bf References}
\bibliography{opint}
\bibliographystyle{alpha}
% \nocite{*}   % put all database entries into the reference list

\printtoc
\bye
