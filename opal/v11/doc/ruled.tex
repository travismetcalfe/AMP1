% file: ruled.tex            TeXsis                  version 2.14
% :: $Header: /usr2/myers/texsis/RCS/ruled.tex,v 1.7 90/08/05 17:59:52 paige Exp Locker: paige $
%======================================================================*
%  RULED TABLES.                                E. Myers and F.E. Paige
%
%       The following macros will typeset tables with vertical and
% horizontal rules. The syntax is similar to Ray Cowan's TABLES.TEX,
% but the macros have been completely rewritten. Vertical and horizontal 
% rules are drawn by macros using vertical spacing controlled by struts.
% Usage:
%       \ruledtable
%       <item> <tab> ... <item> <cr>
%       ...
%       <item> <tab> ... <item> \endruledtable
% where <item> is an element of the table (every row must have exactly 
% the same number of items, although some may be blank), <tab> is any
% one of
%     &                 for no vertical rule
%     |  [or \vb]       for vertical rule
%     \| [or \Vb]       for thick vertical rule
%     \dbl              for double vertical rule
% and <cr> is any one of
%     \nr               for no horizontal rule
%     \cr               for horizontal rule
%     \CR [or \crthick] for thick horizontal rule
% The last <cr> must be omitted in favour of the \endruledtable
%
%  For horizontal rules across only some columns use
%       ... \nr
%       \crule | \cskip & \Crule ... \crpart
% where \crpart is a special \cr for the partial rule line and
%       \cskip          no rule for column
%       \crule          horizontal rule for column
%       \Crule          thick horizontal rule for column
% The number of columns must be identical to the other rows.
%
% To change the appearance of the table, modify \TableItem.
% Several modifications are built in:
%       \LeftJustifyTables      left justify each column
%       \RightJustifyTables     right justify each item
%       \NoJustifyTables        center each item (default)
%       \tablespace             horizontal spacing around each item 
%                               (default is \quad)
%       \TightTables            use \ horizontal spacing
%       \LooseTables            use \quad horizontal spacing (default)
%
% Use \bigitem to expand the vertical strut for high or deep items. 
%
%       These macros have been reorganized and substantially revised from
% version 2.13. The following functional changes have been made:
%       (1) The @ signs have been removed from the names of the macros
% used to save the Plain \cr, \span, and & so that they can be used to
% modify the preamble. See above.
%       (2) Macros for partial horizontal rules have been added.
%       (3) \tabskip glue has been added so that \tablewidth works
% correctly -- the table stretches uniformly over all columns.
%       (4) \linecount has been moved to the \cr macros. It produces a
% warning if the number of columns is different from that in the
% previous row.
%       (5) Trailing spaces in the table items are added to the output.
% A macro \NoTrailingSpaces inserts an \unskip after each item to remove
% them. But this also removes trailing skips unless they are followed by
% an empty \hbox or something similar. This is done automatically for 
% an \hfill (frequently used to left justify an item).
%       The default is \TrailingSpaces, but you may wish to insert
% \NoTrailingSpaces in your TXSsite.tex file.
%       (6) \para uses \normalbaselines rather than specific pt values 
% for the interline spacing and the final strut.
%--------------------------------------------------*
% (C) Copyright 1990 by Eric Myers and Frank Paige.  
% Permission is granted to use this software by any member the 
% academic community in conjunction with basic research.  This software 
% may not be used or distributed for profit without permission.
% This file is a part of TeXsis (but can be used by itself).
%======================================================================*
% Counters and such:     
\catcode`@=11                                   % @ is a letter here
\catcode`\|=12                                  % make sure | is not active
\catcode`\&=4                                   % and that & is alignment tab

\newcount\ncols         \ncols=\z@              % number of columns in table
\newcount\nrows         \nrows=\z@              % number of rows in table
\newcount\curcol        \curcol=\z@             % current column counter
\let\currow=\nrows                              % current row counter
     
\newdimen\thinsize      \thinsize=0.6pt         % thin rule width
\newdimen\thicksize     \thicksize=0.0pt        % thick rule width
\newdimen\tablewidth    \tablewidth=-\maxdimen  % start ``turned off''
\newdimen\parasize      \parasize=4in           % paragraph size in tables

\newif\iftableinfo      \tableinfotrue          % report rows and columns? Yes
\newif\ifcentertables   \centertablestrue       % center tables? Yes
\def\centeredtables{\centertablestrue}%
\def\noncenteredtables{\centertablesfalse}%
\def\nocenteredtables{\centertablesfalse}%      % synonym !
     
\let\plaincr=\cr                        % save real \cr
\let\plainspan=\span                    % save real \span
\let\plaintab=&                         % save real alignment tab &
\def\ampersand{\char`\&}%               % to print `&' in text
\let\lparen=(                           % save left paren     
\let\NX=\noexpand                       % shorthand for \noexpand is \NX

%---------------------------------------*
% Main macros:
%       \ruledtable <table stuff> \endruledtable turns on the ruled
% table definitions and makes the table with \halign. It handles all
% the control stuff; the real work is done by \@RuledTable.
     
\def\ruledtable{\relax                          % make ruled table
    \@BeginRuledTable                           % initialize table
    \@RuledTable}%                              % now process table body

%  \@BeginRuledTable does all the work of setting things up before
% we read in the body of the table

\def\@BeginRuledTable{%                         % initialize table
   \ncols=0\nrows=0                             % reset row and column count
   \begingroup                                  % keep the following local
    \offinterlineskip                           % so vrules touch
    \def~{\phantom{0}}%                         % ~ is phantom digit
    \def\span{\plainspan\omit\relax\colcount\plainspan}%  \span USER columns
    \let\cr=\crrule                             % \cr gives a \tablerule
    \let\CR=\crthick                            % \CR gives a \thickrule
    \let\nr=\crnorule                           % \nr give no rule
    \let\|=\Vb                                  % thick vrule between columns
    \def\hfill{\hskip0pt plus1fill\hbox{}}%     % so it is not removed
% 
% support old \tablestrut in place of \tstrut if it exists
%
    \ifx\tablestrut\undefined\relax             % if not defined, no problem
    \else\let\tstrut=\tablestrut\fi             % use \tablestrut
%
    \catcode`\|=13 \catcode`\&=13\relax         % make | and & active
    \TableActive                                % | and & get active \def`s
    \curcol=1                                   % reset column count
%
% define \Halign to do an \halign with or without a width
%
    \ifdim\tablewidth>-\maxdimen\relax          %
      \edef\@Halign{\NX\halign to \NX\tablewidth\NX\bgroup\TablePreamble}% 
      \tabskip=0pt plus 1fil                    % let things stretch
    \else                                       %
      \edef\@Halign{\NX\halign\NX\bgroup\TablePreamble}% 
      \tabskip=0pt                              % no stretch between columns
    \fi                                         %
%
% center text if needed
%
    \ifcentertables                             % should table be centered?
       \ifhmode\vskip 0pt\fi                    % yes: force vertical mode
       \line\bgroup\hss                         % center across page
    \else\hbox\bgroup                           % else: just put in \hbox
    \fi}%                                       % end of \@BeginRuledTable


%       \@RuledTable builds the table with \@Halign and getting the
% table body text as its argument.

\long\def\@RuledTable#1\endruledtable{%         % ruled table alignment
   \vrule width\thicksize                       % thick rule on side
     \vbox{\@Halign                             % then do \halign
       \thickrule                               % thick rule on top
       #1\killspace                             % body of table
       \tstrut                                  % vertical strut for last line
       \linecount                               % count this line
       \plaincr\thickrule                       % \cr, thick rule on bottom
     \egroup}%                                  % end of \halign and \vbox
   \vrule width\thicksize                       % thick rule on side, end \hbox
   \ifcentertables\hss\fi\egroup                % finish table centering
  \endgroup                                     % end group from \ruledtable
  \global\tablewidth=-\maxdimen                 %   and reset width
  \iftableinfo                                  % report rows and columns
      \immediate\write16{[Nrows=\the\nrows, Ncols=\the\ncols]}%
   \fi}%                                        % end of \@RuledTable
     
%---------------------------------------*
% Preamble and item macros:
%       \TablePreamble is the preamble for the \halign in \ruledtable. 
%       To change how each item is processed change \TableItem.  
%       To make a more complicated table you can change \TablePreamble,
% but if you do so use the following substitutions in a standard \halign 
% preamble:
%    for  &     use     \plaintab
%    for  \cr   use     \plaincr
%    for  #     use     ##
%    for  ##    use     ####

\def\TablePreamble{%                    % \ruledtable preamble
   \TableItem{####}%                    % the first item
   \plaintab\plaintab                   % && means repeat this
   \TableItem{####}%                    % the subsequent items
   \plaincr}%                           % end of preamble


%       \TableItem contains glue or spacing around the item

\def\@TableItem#1{%                     % centers item in ruled table
   \hfil\tablespace                     % left glue
   #1\killspace%                        % item
   \tablespace\hfil                     % right glue
    }%

\def\@tableright#1{%                    % right justifies item in ruled table
   \hfil\tablespace\relax               % left glue
   #1\killspace%                        % item
   \tablespace\relax}%                  % right glue

\def\@tableleft#1{%                     % left justifies item in ruled table
   \tablespace\relax                    % left glue
   #1\killspace%                        % item
   \tablespace\hfil}%                   % right glue

\let\TableItem=\@TableItem              % default is centered
     
\def\RightJustifyTables{\let\TableItem=\@tableright}%   % to right justify
\def\LeftJustifyTables{\let\TableItem=\@tableleft}%     % to left justify
\def\NoJustifyTables{\let\TableItem=\@TableItem}%       % to center


%       \TightTables makes the spacing of a table as tight as possible.
% \LooseTables, the default, gives more pleasing spacing.

\def\LooseTables{\let\tablespace=\quad}%        % table spacing is \quad
\def\TightTables{\let\tablespace=\space}%       % table spacing is space
\LooseTables                                    % default is \quad


%       \TrailingSpaces, the default, allows trailing spaces and glue
% to change the appearance of a table. \NoTrailingSpaces removes all
% trailing spaces and glue before inserting the table glue.

\def\TrailingSpaces{\let\killspace=\relax}      % do not kill trailing space
\def\NoTrailingSpaces{\let\killspace=\unskip}   % kill trailing space
\TrailingSpaces                                 % default

%---------------------------------------*
% Table Height and Width:
%
%  Normally tables are set to their natural width.  If \tablewidth
%  has been set then we set the table to that width instead,
%  but only for the next table.  Then \tablewidth is turned off

%      \setRuledStrut sets up the vertical strut \tstrut with
% the appropriate dimensions to hold up one line of a ruled table. 

\def\setRuledStrut{% sets interlines spacing for ruled tables
   \dimen@=\baselineskip                        % \dimen@ = extra space
   \advance\dimen@ by-\normalbaselineskip       % between lines
   \ifdim\dimen@<.5ex \dimen@=.5ex\fi           % minimum space
   \setbox0=\hbox{\lparen}%                     % get character size
   \dimen1=\dimen@ \advance\dimen1 by \ht0      % space above line
   \dimen2=\dimen@ \advance\dimen2 by \dp0      % space below line
   \def\tstrut{\vrule height\dimen1 depth\dimen2 width\z@}%
   }%

\def\tstrut{\vrule height 3.1ex depth 1.2ex width 0pt}%  default


%      \tstrut does not produce the correct spacing if the entry in
% the table is too high. The following constructs a strut higher than its
% argument and then prints the argument. The minimum space value should
% be the same as in \setRuledStrut. 

\def\bigitem#1{%                                % larger table entry
   \dimen@=\baselineskip                        % \dimen@ = extra space
   \advance\dimen@ by-\normalbaselineskip       % between lines
   \ifdim\dimen@<.5ex \dimen@=.5ex\fi           % minimum space
   \setbox0=\hbox{#1}%                          % get argument size
   \dimen1=\dimen@ \advance\dimen1 by \ht0      % space above line
   \dimen2=\dimen@ \advance\dimen2 by \dp0      % space below line
   \vrule height\dimen1 depth\dimen2 width\z@   % make strut to size
   \copy0}%                                     % print argument


%       \vctr{stuff} centers the stuff vertically, so that it can
% appear between two ROWS.
     
\def\vctr#1{\hfil\vbox to 0pt{\vss\hbox{#1}\vss}\hfil}%

%---------------------------------------*
% Vertical rules:
%       \tab, \vb and \Vb are used in a table to separate columns with no
% rule, a thin rule, or a thick rule, respectively.  \nextcolumn{<rule>} skips
% to  the next column and puts the <rule> between columns.  Use this to build
% your own separators.
     
\def\nextcolumn#1{%                             % move to next col.
   \plaintab\omit#1\relax\colcount              % tab, insert #1, count
   \plaintab}%                                  % tab to next user col.
     
\def\tab{%                                      % no rule between columns
   \nextcolumn{\relax}}%                        % count column

\let\novb=\tab                                  % synonym for \tab

\def\vb{%                                       % thin rule between columns
   \nextcolumn{\vrule width\thinsize}}%         % count and rule

\def\Vb{%                                       % THICK rule between columns
   \nextcolumn{\vrule width\thicksize}}%        % count and thick rule

\def\dbl{%                                      % double rule between columns
   \nextcolumn{\vrule width\thinsize            % count and rule
   \hskip\thinsize\vrule width\thinsize}}%      % and skip and rule

%       \TableActive makes | the same as \vb and & the same as \tab so
% these single characters can be used between columns. Here we have to
% make & and | active so we get the active version of the characters,
% hence the temporary change of \catcode.
     
{\catcode`\|=13 \let|0
 \catcode`\&=13 \let&0
 \gdef\TableActive{\let|=\vb \let&=\tab}%
}% end \catcode`s

%---------------------------------------*
% Horizontal rules:
%       These replacements for \cr put a wide \vrule at the end of the
% line and maybe put a rule under the line, then begin the next line
% with the wide \vrule from the preamble.
     

\def\crrule{\killspace                  % \cr plus rule
   \tstrut                              % strut for spacing
   \linecount                           % count this line
   \plaincr\tablerule                   % \cr, regular rule below line
  }%

\def\crthick{\killspace                 % \cr plus thick rule
   \tstrut                              % strut for vertical spacing
   \linecount                           % count this line
   \plaincr\thickrule                   % \cr, rule, begin next line
  }%                
     
\def\crnorule{\killspace                % \cr plus no rule
   \tstrut                              % strut for spacing
   \linecount                           % count this line
   \plaincr                             % \cr, norule, begin next line
   }%
   
\def\crpart{\killspace                  % end partial rule line
   \linecount                           % count this line
   \plaincr}                            % and end it


%       These rules go across the table.
     
\def\tablerule{\noalign{\hrule height\thinsize depth 0pt}}%
\def\thickrule{\noalign{\hrule height\thicksize depth 0pt}}%


%       Rules for individual columns. You must use \cskip in columns
% with no rules to \omit the \TablePreamble.

\def\cskip{\omit\relax}%
\def\crule{\omit\leaders\hrule height\thinsize depth0pt\hfill}%
\def\Crule{\omit\leaders\hrule height\thicksize depth0pt\hfill}%
     
%---------------------------------------*
% Counting macros:     
%       These macros count rows and columns of the table. After the
% first line has been processed \the\ncols is the total number of
% columns in the table, which may be useful. During processing,
% \the\curcol is the number of the current column, while \the\currow is
% the number of the current row.
     

\def\linecount{%                                % count and check row
   \global\advance\nrows by1%                   % advance row counter
   \ifnum\ncols>0%                              % if \ncols set
      \ifnum\curcol=\ncols\relax\else           % if new \ncols different
      \immediate\write16%                       %   then write message
      {\NX\ruledtable warning: Ncols=\the\curcol\space for Nrow=\the\nrows}%
      \fi\fi                                    % end of tests
   \global\ncols=\curcol                        % save columns in \ncols
   \global\curcol=1}                            % reset counter

\def\colcount{\relax                            %
   \global\advance\curcol by 1\relax}%          % advance column counter

%---------------------------------------*
% TEXT TABLES.
%       To put text in a table use \para{<text>}, with \parasize set
%  to the desired width of the text.

\long\def\para#1{%                      % make a paragraph for tables
   \vtop{\hsize=\parasize               % width - set before \para called
   \normalbaselines                     % set normal baseline skips
   \noindent #1\relax                   % insert text
   \vrule width 0pt depth 1.1ex}%       % hold depth of box
}%

%---------------------------------------*
% TABLES.TEX
%   For compatability with Cowan's TABLES.TEX we also allow the
% syntax \begintable ... \endtable, which do the same thing.
%

\def\begintable{\relax                          % make ruled table
    \@BeginRuledTable                           % initialize table
    \@begintable}%                              % now process table body

\long\def\@begintable#1\endtable{%              % ruled table alignment
   \@RuledTable#1\endruledtable}%               % same as \ruledtable

%---------------------------------------*
% Turn off @ as letter before we exit

\catcode`@=12                                   % @ is not a letter

%>>> EOF TXSruled.tex <<<
